\documentclass[a4paper,NoNotes,GeneralMath]{stdmdoc}
\usepackage{pgf}
\usepackage{tikz-cd}
\usepackage{xfrac}
\usetikzlibrary{cd,arrows,automata}
\newcommand{\MCD}{\text{MCD }}
\newcommand{\Lt}{\text{lt }}
\newcommand{\Lc}{\text{lc }}
\newcommand{\BdG}{\text{BdG }}
\newcommand{\Ris}{\text{Ris }}
\newcommand{\srar}{\twoheadrightarrow}
\newcommand{\hrar}{\hookrightarrow}
\newcommand{\Hom}{\text{Hom }}
\newcommand{\coKer}{\text{coKer }}
\newcommand{\gen}[1]{\langle {#1} \rangle}
\newcommand{\lbr}[1]{\ensuremath{{\left( {#1} \right)}}}
\newcommand{\intclos}[2]{\ensuremath{\overline{#1}^{#2}}}
\newcommand{\Frac}{\text{Frac }}
\newcommand{\Spec}{\text{Spec }}
\newcommand{\Max}{\text{Max }}
\newcommand{\Gal}{\text{Gal }}

\begin{document}
\title{Istituzioni di Algebra}

\section*{Estensioni Intere}
\begin{itemize}
\item ({\bf Equivalenze di Intero}) Sia $A \subseteq B$ un'estensione di
  anelli e sia $b \in B$. Allora le seguenti sono equivalenti:
  \begin{enumerate}
  \item $b$ è intero su $A$
  \item $A[b]$ (come sottoanello) è un $A$-modulo finito
  \item $\exists C \subseteq B$ sottoanello tale che $A[b] \subseteq
    C$ e $C$ è un $A$-modulo finito
  \item $\exists M$, $A[b]$-modulo fedele (ovvero $\Ann(M) = 0$) che
    sia finito come $A$-modulo.
  \end{enumerate}
\item ({\bf Hamilton-Cayley}) Se $M$ è un $A$-modulo finito e
  $I \subseteq A$ ideale, $\phi: M \rar M$ mappa di $A$-moduli tale che
  $\phi(M) \subseteq IM$, allora $\exists a_1, \ldots, a_n \in I$ tali
  che $\phi^n + a_1 \phi^{n-1} + \ldots + a_n \Id = 0$ (in
  $\Hom_A(M, M)$)
\item Valgono quindi le seguenti cose:
  \begin{enumerate}
  \item Se $b_1, \ldots, b_n$ sono interi su $A$, allora
    $A[b_1, \ldots, b_n]$ è un $A$-modulo finito
  \item $\intclos{A}{B} = \{ b \in B | b \text{ intero su } A \}$ è un
    sottoanello di $B$.
  \item ({\bf Transitività Integrale}) Se $B$ è intero su $A$ e $C$ è
    intero su $B$, allora $C$ è intero su $A$
  \item ({\bf Transitività Finita}) Se $B$ è finito su $A$ e $C$ è
    finito su $B$, allora $C$ è finito su $A$
  \item ({\bf Idempotenza della Chiusura Integrale}) Sia $A \subseteq
    B$ e $C = \intclos{A}{B}$. Allora $\intclos{C}{B} = C$
  \end{enumerate}
\item ({\bf Stabilità per Localizzazione e Quoziente}) Sia
  $A \subseteq B$ intera, $S$ parte moltiplicativa di $A$ e
  $I \subseteq B$ ideale. Allora si ha che:
  \begin{enumerate}
  \item $S^{-1}A \rar S^{-1}B$ è intera
  \item $\sfrac{A}{I^c} \rar \sfrac{B}{I}$ è intera
  \end{enumerate}
\item ({\bf Relazioni con estensioni di campi}) Supponiamo $A$ dominio e
  $K = \Frac(A)$ suo campo delle frazioni e consideriamo
  $A \subseteq K \subseteq L$ dove $\sfrac{L}{K}$ è algebrica. Definiamo
  $B = \intclos{A}{L}$.
  \begin{enumerate}
  \item Sia $x \in L$ e $\mu_x$ suo polinomio minimo su $K$. Se
    $\mu_x \in A[t]$ allora $x$ è intero su $A$.
  \item Se $A$ è normale vale anche il viceversa, ovvero si ha $x$ è
    intero su $A$ $\sse$ $\mu_x \in A[t]$.
  \end{enumerate}
\item ({\bf UFD $\implies$ normale}) Se $A$ è un UFD allora è normale.
\item ({\bf estensioni intere di campi}) Sia $A \subseteq B$
  un'estensione intera di domini. Allora $A$ è un campo $\sse$ $B$ è un
  campo. Ne segue che:
  \begin{enumerate}
  \item Sia $\kq \subseteq B$ un ideale. Allora $\kq$ è massimale $\sse$
    $\kq^c$ è massimale
  \item Se prendo $\kp$ primo di $A$ e $\kq_1, \kq_2 \in \Spec B$ tali
    che $\kq_1^c = \kq_2^c = \kq$ e $\kq_1 \subseteq \kq_2$ allora vale
    che $\kq_1 = \kq_2$.
  \end{enumerate}
\item ({\bf Lying Over}) Se $A \subseteq B$ è intera, allora $\forall
  \kp \in \Spec A \quad \exists \kq \in \Spec B$ tale che $\kq^c = \kp$,
  ovvero $f^*: \Spec B \rar \Spec A$ è surgettiva.
\item ({\bf Going Up}) Se $A \subseteq B$ è intera, allora ha la
  proprietà del going up, ovvero se
  $\forall \kp_1 \subseteq \kp_2 \subseteq A$ primi e
  $\forall \kq_1 \subseteq B$ primo tale che $\kq_1^c = \kp_1$ allora
  $\exists \kq_2 \supseteq \kq_1$ primo tale che $\kq_2^c = \kp_2$
\item ({\bf Going Down}) $A \subseteq B$ estensione intera di domini,
  con $A$ normale allora vale la proprietà del going down, ovvero
  $\forall \kp_1 \subseteq \kp_2 \subseteq A$ primi e
  $\forall \kq_2 \subseteq B$ primo tale che $\kq_2 \cap A = \kp_2$ si
  ha che $\exists \kq_1 \subseteq \kq_2$ primo tale che
  $\kq_1 \cap A = \kp_1$. \newline
  Da notare che serve sia la condizione di dominio che la normalità di
  $A$ per far funzionare tutto ciò.
\item Se $A$ è un dominio normale, $K = \Frac A$ ed $\sfrac{L}{K}$ è
  un'estensione algebrica di campi, $B = \intclos{A}{L}$,
  $I \subseteq A$ ideale, ed $x \in L$ allora $x$ è intero su $I$ $\sse$
  $\mu_{\sfrac{L}{K}, x} \in \sqrt{I}[t]$
\item ({\bf Simil-Galois}) Sia $A$ normale e $K = \Frac A \subseteq L$
  con $\sfrac{L}{K}$ estensione di Galois finita e $B = \intclos{A}{L}$
  e sia $G = \Gal \sfrac{L}{K}$. Allora si ha che:
  \begin{enumerate}
  \item $g(B) \subseteq B \quad \forall g \in G$
  \item Fissato un primo $\kp \in \Spec A$ si ha
    $\cF_\kp = \{ \kq \in \Spec B | \kq \cap A = \kp \}$
    Allora $G$ agisce transitivamente su $\cF_\kp$
  \end{enumerate}
\item ({\bf Chiusura integrale di $A[x]$}) Sia $A \subseteq B$ e sia
  $C = \intclos{A}{B}$. Allora la chiusura integrale di $A[x]$ in $B[x]$
  è $C[x]$.
\item ({\bf Interezza e Nötherianità}) $A \subseteq B$ con $A$
  Nötheriano e $B$ finito su $A$. Allora $B$ è Nötheriano come
  anello. \newline
  Attenzione che $A \subseteq B$ con $A$ Nötheriano ed estensione intera
  NON implica che $B$ sia Nötheriano.
\item Sia $A$ dominio, $K = \Frac A$, $\sfrac{L}{K}$ un'estensione
  finita di campi e sia $B = \intclos{A}{L}$. Allora si ha:
  \begin{enumerate}
  \item È FALSO che se $A$ è Nötheriano allora $B$ lo sia.
  \item Se $A$ è normale ed $\sfrac{L}{K}$ è un'estensione separabile
    allora si ha che se $A$ è Nötheriano allora $B$ diventa un
    $A$-modulo finito e quindi è Nötheriano.
  \end{enumerate}
\item $A$ noetheriano e dominio, con $K = \Frac A \subseteq L$ campi e
  vorremmo poter dire qualcosa anche se $A$ non è normale. Ci sono due
  casi significativi nei quali si ha che in queste ipotesi
  $\intclos{A}{K}$ (la normalizzazione di $A$) è Nötheriana:
  \begin{enumerate}
  \item $\Dim A = 1$
  \item $A$ è una $K$-algebra finitamente generata
  \end{enumerate}
\item ({\bf Normalizzazione di Nöther}) $A$ $K$-algebra f.g. allora
  $\exists x_1, \ldots, x_n \in A$ algebricamente indipendenti tali che
  $A$ è finita (come modulo) su $K[x_1, \ldots, x_n]$
  

\item ({\bf Lemmi generici e fatti vari}) Le seguenti cose valgono:
  \begin{enumerate}
  \item $E$ campo e sia $A$ una $E$-algebra finitamente generata. Allora
    $\forall I \subseteq A$ si ha che
    $$\sqrt{I} = \bigcap_{\km \text{ massimali } \quad \km \supseteq I} \km$$
  \item $A$ e $B$ due $K$-algebre finitamente generate con $K$ campo
    algebricamente chiuso ed $A$ dominio. Se $f^*: \Max B \rar \Max A$ è
    suriettiva allora $f$ è iniettiva.
  \item Per un modulo sono proprietà locali (e massimali) essere
    piatto, essere normale, essere nullo.
  \item Per un modulo NON sono proprietà locali essere Noetheriano,
    essere dominio.
  \item Per una sequenza di moduli essere esatta in un punto è una
    proprietà locale (e massimale)
  \end{enumerate}
\end{itemize}

\end{document}

