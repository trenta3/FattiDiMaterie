\usepackage{titlesec} % Per modificare i titoli delle sezioni
\usepackage{fancyhdr} % Per modificare l'intestazione ed il piè di pagina
\usepackage{bm}
\usepackage{bbm}
%\usepackage{mbboard}
\usepackage[utf8]{inputenc}
\usepackage[italian]{babel}
\usepackage{amsmath}
\usepackage{amssymb}
\usepackage{amsthm} % Per definire \newtheorem

\usepackage{xifthen}
\usepackage{xparse}
\usepackage{etoolbox}% http://ctan.org/pkg/etoolbox
\usepackage[linktoc=all]{hyperref} % Per i collegamenti ipertestuali. [linktoc=all] mette i link nell'indice
\usepackage{color} % Per i comandi \textcolor{'colore'}{'testo da colorare'} e \color{'colore'}
\usepackage{hhline} % Per fare tabelle variegate

% https://it.sharelatex.com/blog/2011/03/27/how-to-write-a-latex-class-file-and-design-your-own-cv.html
%mathbb mathcal mathfrak e mathbm per le lettere dell'alfabeto e anche mathbb per quelle greche
\def\mydeflett#1{\expandafter\def\csname bb#1\endcsname{\mathbb{#1}}
		\expandafter\def\csname c#1\endcsname{\mathcal{#1}}
		\expandafter\def\csname k#1\endcsname{\mathfrak{#1}}
		\expandafter\def\csname bl#1\endcsname{\mathbf{#1}}}
\def\mydefalllett#1{\ifx#1\mydefalllett\else\mydeflett#1\expandafter\mydefalllett\fi}
\mydefalllett ABCDEFGHIJKLMNOPQRSTUVWXYZ\mydefalllett

\def\mydeffrakmath#1{\expandafter\def\csname k#1\endcsname{\mathfrak{#1}}}
\def\mydefallfrak#1{\ifx#1\mydefallfrak\else\mydeffrakmath#1\expandafter\mydefallfrak\fi}
\mydefallfrak abcdefghijklmnopqrstuvwxyz\mydefallfrak

\def\mydefgreek#1{\expandafter\def\csname bl#1\endcsname{\text{\boldmath$\mathbf{\csname #1\endcsname}$}}}
\def\mydefallgreek#1{\ifx\mydefallgreek#1\else\mydefgreek{#1}%
   \lowercase{\mydefgreek{#1}}\expandafter\mydefallgreek\fi}
\mydefallgreek {Gamma}{Delta}{Theta}{Lambda}{Xi}{Pi}{Sigma}{Upsilon}{Phi}{Varphi}{Psi}{Omega}{alpha}{beta}{gamma}{delta}{epsilon}{varepsilon}{zeta}{eta}{theta}{iota}{kappa}{lambda}{mu}{nu}{xi}{omicron}{pi}{rho}{sigma}{tau}{upsilon}{phi}{varphi}{chi}{psi}{omega}\mydefallgreek

\NewDocumentCommand{\de}{gg}{
	\IfNoValueTF{#1}
		{\text{ d}}
		{\IfNoValueTF{#2}	{\text{ d}#1}
			{\frac{\text{d}#1}{\text{d}#2}}
	}
}

\NewDocumentCommand{\dpar}{gg}{
	\IfNoValueTF{#1}
		{\partial}
		{\IfNoValueTF{#2}	{\partial_{#1}}
			{\frac{\partial {#1}}{\partial {#2}}}
	}
}


%nuovi comandi per svariate cose
\newcommand{\sse}{\Leftrightarrow}
\newcommand{\Rar}{\Rightarrow}
\newcommand{\rar}{\rightarrow}
\newcommand{\ol}[1]{\overline{#1}}
\newcommand{\ot}[1]{\widetilde{#1}}
\newcommand{\oc}[1]{\widehat{#1}}
\newcommand{\tc}{\mbox{ t.c. }}

\newcommand{\norma}[1]{\mid\mid #1 \mid\mid}
\newcommand{\abs}[1]{\mid #1 \mid}
\newcommand{\scal}[2]{\langle #1 \mid #2 \rangle}
\newcommand{\floor}[1]{\lfloor #1 \rfloor}

\newcommand{\Ker}{\mbox{Ker } }
\newcommand{\Deg}{\mbox{deg }}
\newcommand{\Det}{\mbox{det }}
\newcommand{\Dim}{\mbox{dim }}
\newcommand{\End}{\mbox{End }}
\newcommand{\Rad}{\mbox{Rad }}
\newcommand{\Ann}{\mbox{Ann }}
\newcommand{\Sp}{\mbox{Sp }}
\newcommand{\Rk}{\mbox{rk }}
\newcommand{\Tr}{\mbox{tr}}
\newcommand{\GL}{\mbox{GL}}
\newcommand{\Isom}{\mbox{Isom}}
\newcommand{\Fix}{\mbox{Fix }}
\newcommand{\Giac}{\mbox{Giac }}
\newcommand{\Ort}{\mbox{O}}
\newcommand{\Aff}{\mbox{Aff }}
\newcommand{\Supp}{\mbox{Supp }}
\newcommand{\Span}{\mbox{Span }}
\newcommand{\Symm}{\mbox{Sym }}
\newcommand{\Asymm}{\mbox{Asym }}
\newcommand{\Img}{\mbox{Im }}
\newcommand{\Id}{\mbox{id}}
\newcommand{\PS}{\mbox{PS }}
\newcommand{\Mtr}{\mathfrak{m}}
\newcommand{\fucknullset}{\{0\}}
\newcommand{\TODO}{{\LARGE\bf TO DO}}

% -- Comandi propri di QINFO --
\newcommand{\ket}[1]{\left|{#1}\right\rangle}
\newcommand{\bra}[1]{\left\langle{#1}\right|}
\newcommand{\braket}[2]{\left\langle{#1}\middle|{#2}\right\rangle}
\newcommand{\ketbra}[2]{\left|{#1}\middle\rangle\ \middle\langle{#2}\right|}
\newcommand{\braaket}[3]{\left\langle{#1}\middle|{#2}\middle|{#3}\right\rangle}

\newcommand{\lrg}[1]{\left\{{#1}\right\}}
\newcommand{\lrq}[1]{\left[{#1}\right]}
\newcommand{\lrt}[1]{\left({#1}\right)}

\sloppy

\setlength\parskip{0.5em}
\setlength\parindent{0cm}
\linespread{1.05}

