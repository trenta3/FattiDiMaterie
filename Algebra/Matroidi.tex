\documentclass[a4paper,NoNotes,GeneralMath]{stdmdoc}

\begin{document}
	\title{Piccola Teoria dei Matroidi}
	
	\section*{Definizioni}
	\subsection*{Matroide}
	Un Matroide è una coppia di insiemi $(E, \cI)$ tali che $\cI \subseteq \cP(E)$ e $\cI$ ha le seguenti proprietà:
		\begin{itemize}
			\item {\bf (I1)} $\cI$ è non vuoto
			\item {\bf (I2 - Ereditarietà)} $B \in \cI$ e $A \subseteq B \implies A \in \cI$
			\item {\bf (I3 - Augmentation)} $X, Y \in \cI$ e $\mid X \mid = \mid Y \mid + 1$ allora esiste un elemento $x \in X \setminus Y$ tale che $Y \cup \{ x \} \in \cI$
			\item {\bf (I4 - Extension)} $\{B_\lambda\}_{\lambda \in \Lambda}$ una catena di elementi di $\cI$ ordinata per inclusione. Allora $\cup_\lambda B_\lambda \in \cI$
		\end{itemize}
	L'idea sarebbe quella di catturare la proprietà dell'"Essere Indipendenti", ad esempio si può pensare ad $\cI$ come l'insieme dei sottoinsiemi linearmente indipendenti di uno spazio vettoriale $V = E$ oppure come gli elementi algebricamente indipendenti di un'estensione di campi. \\
	Con una semplice induzione si nota che la proprietà di Augmentation vale anche nella seguente forma un po' più generale (e che utilizzeremo in seguito): $X, Y \in \cI$ e $\mid X \mid \ge \mid Y \mid + n$ allora esistono $n$ elementi $x_1, \ldots, x_n \in X \setminus Y$ tali che $Y \cup \{x_1, \ldots, x_n\} \in \cI$.
	
	\subsection*{Base}
	Si dice {\it Base} un qualunque elemento di $\cI$ massimale per inclusione. Esistono sempre (basta usare in maniera classica Zorn sfruttando I4 per dire che una catena di tizi indipendenti è ancora indipendente)
	
	\subsection*{Duale}
	
	\subsection*{Somma Diretta}
	
	\section*{Teoremini}
	\subsection*{}

\end{document}
