\IfFileExists{stdmdoc.cls}
	{} %Do nothing
	{\immediate\write18{wget https://github.com/trenta3/stdmdoc/raw/master/stdmdoc.cls}} %else download file
\documentclass[a4paper, NoNotes, GeneralMath]{stdmdoc}

\newcommand{\E}[1]{\mathbb{E}[{#1}]}
\newcommand{\Var}[1]{\mbox{Var }[{#1}]}
\newcommand{\va}{\mbox{ v.a. }}
\newcommand{\iid}{\mbox{ i.i.d. }}

\begin{document}
	\title{Fatti di EPS}
	\autodate

	\section*{Cose}
	\begin{itemize}
		\item Se vari eventi $A_1, \ldots, A_n$ sono indipendenti, allora anche i loro complementari $A_1^C, \ldots, A_n^C$ sono indipendenti
		\item {\bf Legge dei grandi numeri}: $X_1, \ldots$ successione di $\va \iid$, $S_n := X_1 + \ldots + X_n$. Allora vale $\forall \varepsilon > 0 \qquad \lim_{n\rar +\infty} P\left\{ \left| \frac{S_n}{n} - p \right| > \varepsilon \right\} = 0$
	\end{itemize}

	\section*{Funzioni Generatrici}
	Si indica con $G_X(t)$ la funzione generatrice della variabile aleatoria $X$
	\begin{itemize}
		\item $G_X(t) = G_Y(t) \sse X$ e $Y$ sono equidistribuite
		\item Se $X$ e $Y$ sono indipendenti, allora $G_{X+Y}(t) = G_X(t)\cdot G_Y(t)$
		\item $\E{X} = \lim_{t\rar 1^{-}} G_X'(t)$
		\item $\E{X(X-1)} = \lim_{t\rar 1^{-}} G_X''(t)$
	\end{itemize}
	

	\section*{Tabella delle Distribuzioni di Probabilità Discrete}
	\begin{tabular}{lccccl}
		{\bf Nome} & {\bf $p(k) = P\{X = k\}$} & {\bf $G(t)$ generatrice} & {\bf $\E{X}$} & {\bf $\Var{X}$} & {\bf Condizioni} \\
		Geometrica & $(1-p)^{k-1}p$ & $\frac{tp}{1-t(1-p)}$ & $\frac{1}{p}$ & $\frac{1-p}{p^2}$ & $p \in (0,1)$, $k \in \bbN$ \\
		Binomiale & ${n \choose k}p^k(1-p)^{n-k}$ & $[1+p(t-1)]^n$ & $np$ & $np(1-p+np)$ & $p \in (0,1)$, $k \in \{0, \ldots, n\}$ \\
		Poisson & $e^{-\lambda}\frac{\lambda^n}{n!}$ & $e^{\lambda (t-1)}$ & $\lambda$ & $\lambda(\lambda +1)$ & $\lambda > 0$, $n \in \bbN$ \\
		Binomiale negativa & & & & & \\
		Ipergeometrica & & & & & \\	
	\end{tabular} \vskip 1em 
	La binomiale negativa : si ripete in condizioni di indipendenza un esperimento che ha probabilità p di successo fino a che questo si realizza k volte. La variabile conta il numero di tentativi che è stato necessario effetuare. \\ L'ipergeometrica: Consideriamo un'urna contentente r sfere rosse e b sfere bianche , ed in essa compiamo n estrazioni senza reimussolamento. Consideriamo la v.a. che conta il numero di sfere rosse che sono state estratte. \\

	\section*{Tabella delle Distribuzioni di Probabilità Continue}
	\begin{tabular}{lccccl}
	{\bf Nome} & {\bf $f(x)$ densità} & {\bf $F(x)$ cumulativa} & {\bf $\E{X}$} & {\bf $\Var{X}$} & {\bf Condizioni} \\
	Esponenziale & $\lambda e^{-\lambda x}$ & $1 - e^{-\lambda x}$ & $\frac{1}{\lambda}$ & $\frac{1}{\lambda^2}$ & $\lambda > 0$, $x \in \bbR^{+}$ \\
	
	\end{tabular} \vskip 1em
\end{document}
