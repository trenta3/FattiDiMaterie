\documentclass[a4paper,NoNotes,GeneralMath]{stdmdoc}
\usepackage{tikz}
\usetikzlibrary{arrows,automata}
\newcommand{\Enunciato}{\vskip 0.05cm \noindent \textbf{Enunciato} \\ }
\renewcommand{\Dimostrazione}{\vskip 0.05cm \noindent \textbf{Dimostrazione} \\ }
\newcommand{\gen}{\text{Gen }}
\newcommand{\MCD}{\text{MCD }}
\newcommand{\Lead}{\text{Lead }}
\renewcommand{\bar}{\overline}
\newcommand{\ims}{\hookrightarrow}

\begin{document}
	\title{Teoria dei Polinomi}
	Argomenti trattati: polinomi in una o più variabili, polinomi simmetrici, polinomi omogenei, teoria del risultante.

	\section*{Notazione e Convenzioni}
	All'interno della presente trattazione adottiamo le seguenti convenzioni:
	\begin{itemize}
		\item Quando non diversamente specificato assumiamo che $R$ sia un anello commutativo unitario ed anche dominio d'integrità. In particolare il fatto che $R$ sia ID, ci permette di dire che, se $f,g \in R[x]$ allora $\Deg(fg) = \Deg(f) + \Deg(g)$, cosa che useremo abbastanza spesso.
		\item Tutte le sommatorie che compaiono si intendono finite
		\item Con $a \mid_S b$ intendiamo che $\exists s \in S \tc b = as$
	\end{itemize} \vskip 0.2cm

	Ed useremo la seguente notazione:
	\begin{itemize}
		\item Indichiamo con $Q_R$ il campo delle frazioni su $R$
		\item $f'(x)$ indica la derivata formale di $f(x)$, ovvero se $f(x) = \sum_i a_i x^i$ definiamo $f'(x) = \sum_i (i\star a_i) x^{i-1}$, dove $\star: \bbN \times R \rar R$ è tale che $\star(n, r) = \underbrace{r + r + \ldots + r}{n \text{ volte}}$.
		\item Con $\bbP_R$ indichiamo l'insieme dei primi in $R$
	\end{itemize}

	\section*{Polinomi in una variabile}

	\subsection{Teorema di Ruffini}
	\Enunciato Sia $f(x) \in R[x]$. Allora $f(\alpha)=0 \sse (x-\alpha) \mid_R f(x)$
	\Dimostrazione \frdx $\quad $ Notiamo che possiamo effettuare la solita divisione euclidea tra $f(x)$ e $(x-\alpha)$ restando ad ogni passaggio in $R[x]$ in quanto $x-\alpha$ è monico. Allora si ha $\exists q(x), r(x) \in R[x] \tc f(x) = q(x)(x-\alpha) + r(x)$, con $\Deg r < 1 \opp r = 0$. Valutando in $\alpha$ si ha $0 = f(\alpha) = r(\alpha) \implies r = 0$ perché $r$ ha al più grado $0$. \\
		\frsx $\quad$ Scriviamo $f(x) = (x-\alpha)q(x)$ e valutando in $\alpha$ si ha la tesi.
	
	\subsection{Lemma della derivata e molteplicità delle radici}
	\Enunciato $f(x) \in R[x]$. Allora $(x-\alpha)^2 \mid_R f(x) \sse f(\alpha) = 0 \text{ e } f'(\alpha) = 0$.
	\Dimostrazione \frdx $\quad$ $f(x) = (x-\alpha)^2g(x) \implies f(\alpha) = (\alpha-\alpha)^2g(\alpha) = 0$ e $f'(\alpha) = (2(x-\alpha)g(x) + (x-\alpha)^2 g'(x))\mid_{x = \alpha} = 0$ \\
		\frsx $\quad$ Dal teorema di Ruffini sappiamo che $f(\alpha) = 0 \implies f(x) = xh(x)$. Allora $f'(x) = h(x) + (x-\alpha) h'(x)$ e $0 = f'(\alpha) = h(\alpha)$ quindi, ancora per Ruffini, abbiamo $(x-\alpha) \mid h(x)$, ovvero $(x-\alpha)^2 \mid f(x)$

	\subsection{Massimo numero di radici del polinomio}
	\Enunciato $f(x) \neq 0 \in R[x], \Deg f = n$. Allora $f(x)$ ha al più $n$ radici in $R$.
	\Dimostrazione Ogni volta che troviamo una radice $\alpha$ di $f$, possiamo dire $f(x) = (x-a)g(x)$ e abbiamo che $\Deg g = \Deg f - 1$, da cui la tesi.

	\subsection{Teorema delle Radici in $Q_R$}
	\Enunciato Sia $R$ GCD, $f(x) \in R[x]$, $\Deg f = n$, $f(x) = \sum_i a_i x^i$, $\alpha \in Q_R$ una sua radice. Allora, detti $p, q \in R \tc \alpha = \frac{p}{q}$, si ha che $p \mid a_0$ e $q \mid a_n$.
	\Dimostrazione Sappiamo che $0 = f(\frac{p}{q}) = a_n (\frac{p}{q})^n + \ldots + a_1 \frac{p}{q} + a_0$ e possiamo supporre $\frac{p}{q}$ ridotta ai minimi termini, ovvero con $(p,q) = 1$. Moltiplicando da ambo i lati per $q^n$ si ottiene $0 = a_n p^n + a_{n-1} p^{n-1}q + \ldots + a_1 pq^{n-1} + a_0 q^n$ e notiamo che $q$ divide tutti i termini tranne $a_n p^n$ e $p$ divide tutti i termini tranne $a_0 q^n$, quindi si ha, poiché $q$ e $p$ sono coprimi, $p \mid a_0$ e $q \mid a_n$.

	\subsection{Principio di identità dei Polinomi}
	\Enunciato $f(x) \in R[x]$, $\Deg f = n$, $f(x) = \sum_i a_i x^i$. Supponiamo $\exists \alpha_1, \ldots, \alpha_{n+1}$ $n+1$ radici con molteplicità di $f(x)$. Allora $f(x) \equiv 0$.
	\Dimostrazione Ovvia, segue dal "Massimo numero di radici del polinomio".

	\subsection{Strana Divisibilità}
	\Enunciato $f(x) \in R[x]$, $a, b \in R$. Allora $(b-a) \mid_R (f(b) - f(a))$.
	\Dimostrazione Effettuiamo la divisione di $f(x)$ per $(x-a)$. Si ha $\exists q(x), r(x) \in R[x]$ tali che $f(x) = (x-a)q(x) + r(x)$. Ora valutando in $a$ si ottiene $f(a) = r(a) = r(x)$ (perché $\Deg r \le 0$) e, valutando in $b$ si ha $f(b) = (b-a)q(b) + r(b) = (b-a)q(b) + f(a)$, e sottraendo $f(b)-f(a) = (b-a) q(b)$, quindi $(b-a) \mid_R (f(b)-f(a))$.

	\subsection{Criterio di Irriducibilità di Eisenstein}
	\Enunciato $f(x) = \sum_i a_i x^i \in R[x]$, $\Deg f = n$. Se $\exists p \in \bbP_R \tc p \nmid a_n, \quad p \mid a_0, a_1, \ldots, a_{n-1}, \quad p^2 \nmid a_0$ allora $f(x)$ si può ridurre solo come $\beta \cdot h(x)$ con $\beta \in R$.
	\Dimostrazione Supponiamo $\exists g(x), h(x) \in R[x] \tc f(x) = g(x) \cdot h(x)$. Sia $A = R/(p)$ il dominio d'integrità quoziente (perché $(p)$ è un ideale primo). Allora abbiamo $\bar{f}(x) = \bar{a_n}x^n$. Quindi la fattorizzazione di $\bar{f} = \bar{g} \cdot \bar{h}$ implica $\bar{g}, \bar{h}$ sono monomi (perché altrimenti il prodotto ha più termini di uno siccome $A$ è ID). Allora abbiamo $\bar{g} = \bar{g_s}x^s$, $\bar{h} = \bar{h_r}x^r$, con $\bar{g_s}, \bar{h_r} \neq_A 0$. Quindi $s+r = n$ e se $s \opp r \ge 1$ si ha $p^2 \mid a_0$. Assurdo. Allora WLOG $\Deg g = 0$. Ovvero $f(x) = g_0 \cdot h(x)$.

	\subsection{Irriducibilità per Traslazioni}
	\Enunciato Se $f(x)$ si fattorizza come $g(x)h(x)$, allora anche $f(x+a)$ si fattorizza
	\Dimostrazione Ovvia: $g(x+a)h(x+a) = f(x+a)$ e notiamo che $\Deg g(x+a) = \Deg g(x)$ e $\Deg h(x+a) = \Deg h(x)$. Può essere usato con profitto per poi usare Eisenstein sul polinomio traslato.

	\subsection{Hensel lifting lemma}
	Qui i primi {\bf non} indicano la derivata, ma altri polinomi.
	\Enunciato $I \subseteq R$ ideale. Dati $f,g,h,s,t \in R$ tali che $f \equiv gh \mod I$ e $sg + th \equiv 1 \mod I$ allora $\exists g',h' \in R$ tali che $f \equiv g'h' \mod I^2$, $g' \equiv g \mod I$ e $h' \equiv h \mod I$. Inoltre se $g'$ e $h'$ soddisfano le condizioni precedenti allora si ha anche $s'g'+t'h' \equiv 1 \mod I^2$ per qualche $s' \equiv s \mod I$ e $t' \equiv t \mod I$. $g',h'$ sono unici nel senso che ogni altra soluzione $g^{*}$ e $h^{*}$ che soddisfa le condizioni sopra soddisfa anche $g^{*} \equiv (1+u) g' \mod I^2$ e $h^{*} \equiv (1-u) h' \mod I^2$ per qualche $u \in I$.
	\Dimostrazione Sia $f - gh \equiv e \mod I^2$, verifichiamo che $g' := g + te \mod I^2$ e $h' := h + se \mod I^2$ soddisfano le condizioni $f \equiv g'h' \mod I^2$, $g' \equiv g \mod I$ e $h' \equiv h \mod I$. Ci riferiamo a queste tre condizioni assieme con C. \\
	Per tutti i $g', h'$ che soddisfano C, sia $d := sg' + th' -1 \mod I^2$, verifichiamo che $s' := (1-d) s \mod I^2$ e che $t' := (1-d) t \mod I^2$ soddisfano le condizioni $s'g' + t'h' \equiv 1 \mod I^2$, $s \equiv s' \mod I$ e $t' \equiv t \mod I$. \\
	Supponiamo che $g^{*}, h^{*}$ siano altre soluzioni che soddisfano C. Sia $v := g^{*} - g'$, $w := h^{*} - h'$. La relazione $g^{*}h^{*} \equiv g'h' \mod I^2$ implica $g'w+h'v \equiv 0 \mod I^2$, siccome $v, w \in I$. Allora visto che $s'g'+t'h' \equiv 1 \mod I^2$, moltiplicando entrambi i membri per $v$ otteniamo $(s'v-t'w)g' \equiv v \mod I^2$. Prendendo $u = s'v-t'w \in I$, $g^{*} \equiv (1+u)g' \mod I^2$, in maniera simile $h^{*} = (1-u)h' \mod I^2$.

	\section*{Polinomi in più variabili}
	\subsection{Principio di Identità dei Polinomi}
	\Enunciato $R$ di cardinalità infinita. Se $f \in R[x_1, \ldots, x_n]$ è tale che $\forall a = (a_1, \ldots, a_n) \in R^n \quad f(a) = 0$ allora si ha $f \equiv 0$, ovvero $f$ è il polinomio identicamente nullo.
	\Dimostrazione Mostriamo per induzione sul numero di incognite $n$ che se $f \neq 0$ allora esiste un punto dove $f$ non ha valore nullo. Per $n=1$ l'abbiamo già fatto con l'analogo teorema in una variabile. Mostriamo ora il passo induttivo: supponiamo che $f \in R[x_1, \ldots, x_n][x_{n+1}]$ e chiamiamo $y = x_{n+1}$ per comodità. Allora, ordinando i termini per il loro grado in $y$ si ha $f = y^s (a_0 + a_1 y + \ldots + a_r y^r)$. Prendiamo il punto $\bar{x} \in R^n \tc a_0(\bar{x}) \neq 0$ e valutiamo tutti i polinomi $a_k$ in $\bar{x}$, ottenendo $f(\bar{x}, y) = y^s (u_0 + u_1 y + \ldots + u_r y^r)$ dove $u_j = a_j(\bar{x}) \in R$. Sapendo che ora $g(y) := f(\bar{x}, y) \in R[y]$ è non nullo e che $R$ ha cardinalità infinita so che $\exists q \in R \tc g(q) \neq 0$ allora so che il punto $(\bar{x}, q)$ è tale che $f(\bar{x}, q) \neq 0$. Abbiamo così dimostrato ciò che volevamo.

	\subsection{Nullstellensatz}
	{\bf Lemma delle $K$-algebre} \\
	\Enunciato Dato $K$ un campo, sia $L$ una $K$-algebra finitamente generata su $K$. Se $L$ è anche un campo, allora $L$ è algebrico su $K$
	\Dimostrazione Sia $L = K[\alpha_1, \ldots, \alpha_n]$. Supponiamo per assurdo che $L$ {\bf non} sia algebrico su $K$. Allora $\exists i \tc \alpha_i$ non è algebrico su $K$ (se lo fossero tutti avrei $L/K$ algebrico per torri). Consideriamo quindi $K(\alpha_i) \ims L$ poichè $L$ è un campo. Inoltre abbiamo $K(x) \cong K(\alpha_i)$ perché usando il morfismo che manda $x \mapsto \alpha_i$ otteniamo che ha $\Ker$ banale (altrimenti troviamo $p \in K[x] \tc p(\alpha_i) = 0$ assurdo). Adesso mostriamo che $K(x)$ {\bf non} è finitamente generata come $K$-algebra: supponiamo che lo sia. Allora esistono $\{e_i\}_{i=1}^{r} \subset K(x) \tc \forall f(x) \in K(x) \quad f(x) = \sum_{i}^{\text{finite}} s_i \prod_{j}^{\text{finite}} e_j$ dove $s_i \in K$. Ma, scrivendo $e_i = \frac{a_i(x)}{b_i(x)}$ notiamo che necessariamente si avrebbe che ogni elemento di $K(x)$ può avere al denominatore solo elementi irriducibili che compaiono nella fattorizzazione di almeno uno dei $b_i(x)$, denominatori della base in numero finito. Mostrando ora che esistono infiniti polinomi irriducibili in $K[x]$ terminiamo la dimostrazione, ottenendo un assurdo e dovendo quindi avere che $L/K$ è algebrico. \\
	Supponiamo che esistano solo un numero finito di polinomi irriducibili in $K[x]$. Siano essi $p_1, \ldots, p_m$. Consideriamo allora $S^{*} = \left( \prod_{i=1}^{m} p_i \right) + 1$. Siccome $K[x]$ è PID (e quindi UFD) abbiamo che gli elementi irriducibili sono anche primi, quindi i $(p_i)$ sono ideali primi, ovvero sono anche massimali. O $S^{*}$ è irriducibile, assurdo, oppure $S^{*} = \prod_{j=1}^{m} p_j^{\beta_j}$. Sia $\bar{k} \tc \beta_k \ge 1$ e consideriamo $S^{*} \mod (p_k)$. Otteniamo $0 \equiv \prod_{j=1}^m p_j^{\beta_j} \equiv S^{*} \equiv 1 + \left( \prod_{i=1}^m p_i \right) \equiv 1 \mod (p_k)$ quindi $(p_k) = (1)$ e $p_k$ è invertibile, contro l'ipotesi che fosse irriducibile. Abbiamo quindi l'assurdo voluto. \\
	
	{\bf Nullstellensatz, forma debole} \\
	\Enunciato Sia $K$ un campo algebricamente chiuso. Allora ogni ideale massimale nell'anello di polinomi $R = K[x_1, \ldots, x_n]$ ha la forma $(x_1-a_1, \ldots, x_n-a_n)$ per qualche $a_1, \ldots, a_n \in K$. Come conseguenza, una famiglia di funzioni polinomiali su $K^n$ con nessuno zero in comune genera l'ideale unitario di $R$.
	\Dimostrazione Se $M$ è un ideale massimale di $R$, allora $R/M$ è un campo che è finitamente generato come $K$-algebra. Per il lemma precedente, e poiché $K$ è algebricamente chiuso si ha $R/M \cong K$. Quindi ogni $x_i$ viene mappato in qualche $a_i \in K$ dalla mappa naturale $R \rar R/M \cong K$, quindi $M$ contiene l'ideale $(x_1-a_1, \ldots, x_n-a_n)$. Questo è un ideale massimale, quindi è uguale a $M$. Per quanto riguarda la seconda affermazione, si consideri l'ideale generato da qualche funzione polinomiale data senza zeri in comune. Se stesse in qualche ideale massimale, ovvero $(x_1-a_1, \ldots, x_n-a_n)$, allora tutte le funzioni dovrebbero avere uno zero in $(a_1, \ldots, a_n) \in K^n$, contrariamente alle ipotesi. Siccome non sta in nessun ideale massimale, deve essere tutto $R$. \\

	{\bf Nullstellensatz, forma forte} \\
	\Enunciato Sia $K$ un campo algebricamente chiuso e $g$ e $f_1, \ldots, f_m$ siano membri di $R = K[x_1, \ldots, x_n]$, visti come funzioni polinomiali su $K^n$. Se $g$ si azzera sul luogo degli zeri comuni degli $f_i$, allora qualche potenza di $g$ appartiene all'ideale che generano.
	\Dimostrazione ({\it Rabinowitsch trick}: aggiungiamo un'incognita) I polinomi $f_1, \ldots, f_m$ e $x_{n+1}g-1$ non hanno zeri comuni in $K^{n+1}$, quindi per il Nullstellensatz debole si ha $$ 1 = p_1 f_1 + \ldots + p_m f_m + p_{m+1} (x_{n+1}g-1) $$ dove i $p_i$ sono polinomi in $x_1, \ldots, x_{n+1}$. Prendendo l'immagine di questa equazione attraverso l'omomorfismo $K[x_1, \ldots, x_{n+1}] \rar K(x_1, \ldots, x_n)$ dato da $x_{n+1} \mapsto \frac{1}{g}$ troviamo che $$ 1 = p_1(x_1, \ldots, x_n, \frac{1}{g}) f_1 + \ldots + p_m(x_1, \ldots, x_n, \frac{1}{g}) f_m$$ Moltiplicando ora per la giusta potenza di $g$ per cancellare i denominatori si ha la tesi.

	\section*{Polinomi simmetrici}

	\section*{Polinomi omogenei}
	\subsection{I fattori di polinomi omogenei sono omogenei}

	\section*{Il risultante}

\end{document}
