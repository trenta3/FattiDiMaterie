%\documentclass[a4paper,NoNotes,GeneralMath]{stdmdoc} 
\documentclass[a4paper,10pt,oneside]{math_article}

\title{Lezioni di Algebra 1}
\author{Persone varie ed eventuali}
\date{}


\begin{document}


\maketitle

\cleardoublepage	
  \section{Lezione 1}
    \subsection{Richiami sui gruppi}
      Iniziamo con un ripasso del corso di Aritmetica. Si dice \myname{gruppo} una coppia $(G,\cdot)$ con $e \in G$ elemento neutro, e in cui esiste sempre l'inverso. Un gruppo può essere abeliano se $\forall x,y\in G$ vale $xy=yx$. Di non abeliano conosciamo solo il gruppo delle permutazioni $S(n)$.
      
      Un gruppo si dice \myname{ciclico} se è generato da un solo elemento, i.e. $G = \gen x = \{x^n: n\in \mathbb Z\}$. Si dimostra facilmente che un gruppo ciclico è isomorfo a $\mathbb Z/n\mathbb Z$.
      
      Un gruppo ciclico è ovviamente abeliano ma NON vale il viceversa.
      
      Si dice insieme dei generatori di un gruppo G un insieme $S\subseteq G$ tale che non esistono $H<G$ con $S\subseteq H$, e si indica $\gen S =G$ ($S$ genera $G$). In generale $\gen S$ è l'insieme dei prodotti finiti di elementi di $S$ e dei loro inversi.
      
      \begin{mytheorem}[Teorema di Lagrange]
	Sia $G$ un gruppo con $\abs G = n$, sia $H<G$, con $\abs H = d$. Allora $d\mid n$.
      \end{mytheorem}
      
      \begin{mytheorem}[Teorema di Cauchy]
	Sia $G$ un gruppo con $\abs G = n$ e sia $d\mid n$. Inoltre supponiamo che sia vera almeno una tra
	\begin{itemize}
	  \item $G$ è ciclico;
	  \item $G$ è abeliano;
	  \item $d$ è primo.
	\end{itemize}
	
	Allora esiste $H<G$ con $\abs H=d$.
      \end{mytheorem}
      
      
\end{document}
