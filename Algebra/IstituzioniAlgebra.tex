\documentclass[a4paper,NoNotes,GeneralMath]{stdmdoc}
\usepackage{pgf}
\usepackage{tikz-cd}
\usepackage{xfrac}
\usetikzlibrary{cd,arrows,automata}
\newcommand{\MCD}{\text{MCD }}
\newcommand{\Lt}{\text{lt }}
\newcommand{\Ass}{\text{Ass }}
\newcommand{\cl}{\ell}
\newcommand{\Lc}{\text{lc }}
\newcommand{\pipe}{\,|\,}
\newcommand{\BdG}{\text{BdG }}
\newcommand{\Ris}{\text{Ris }}
\newcommand{\Ht}{\text{ht }}
\newcommand{\Coht}{\text{coht }}
\newcommand{\srar}{\twoheadrightarrow}
\newcommand{\hrar}{\hookrightarrow}
\newcommand{\Hom}{\text{Hom }}
\newcommand{\coKer}{\text{coKer }}
\newcommand{\gen}[1]{\langle {#1} \rangle}
\newcommand{\lbr}[1]{\ensuremath{{\left( {#1} \right)}}}
\newcommand{\intclos}[2]{\ensuremath{\overline{#1}^{#2}}}
\newcommand{\Frac}{\text{Frac }}
\newcommand{\Spec}{\text{Spec }}
\newcommand{\Max}{\text{Max }}
\newcommand{\Gal}{\text{Gal }}
\newcommand{\trdeg}{\text{trdeg }}

\begin{document}
\title{Istituzioni di Algebra}

\section*{Estensioni Intere}
\begin{itemize}
\item ({\bf Equivalenze di Intero}) Sia $A \subseteq B$ un'estensione di
  anelli e sia $b \in B$. Allora le seguenti sono equivalenti:
  \begin{enumerate}
  \item $b$ è intero su $A$
  \item $A[b]$ (come sottoanello) è un $A$-modulo finito
  \item $\exists C \subseteq B$ sottoanello tale che $A[b] \subseteq
    C$ e $C$ è un $A$-modulo finito
  \item $\exists M$, $A[b]$-modulo fedele (ovvero $\Ann(M) = 0$) che
    sia finito come $A$-modulo.
  \end{enumerate}
\item ({\bf Hamilton-Cayley}) Se $M$ è un $A$-modulo finito e
  $I \subseteq A$ ideale, $\phi: M \rar M$ mappa di $A$-moduli tale che
  $\phi(M) \subseteq IM$, allora $\exists a_1, \ldots, a_n \in I$ tali
  che $\phi^n + a_1 \phi^{n-1} + \ldots + a_n \Id = 0$ (in
  $\Hom_A(M, M)$)
\item Valgono quindi le seguenti cose:
  \begin{enumerate}
  \item Se $b_1, \ldots, b_n$ sono interi su $A$, allora
    $A[b_1, \ldots, b_n]$ è un $A$-modulo finito
  \item $\intclos{A}{B} = \{ b \in B \pipe b \text{ intero su } A \}$ è un
    sottoanello di $B$.
  \item ({\bf Transitività Integrale}) Se $B$ è intero su $A$ e $C$ è
    intero su $B$, allora $C$ è intero su $A$
  \item ({\bf Transitività Finita}) Se $B$ è finito su $A$ e $C$ è
    finito su $B$, allora $C$ è finito su $A$
  \item ({\bf Idempotenza della Chiusura Integrale}) Sia $A \subseteq
    B$ e $C = \intclos{A}{B}$. Allora $\intclos{C}{B} = C$
  \end{enumerate}
\item ({\bf Stabilità per Localizzazione e Quoziente}) Sia
  $A \subseteq B$ intera, $S$ parte moltiplicativa di $A$ e
  $I \subseteq B$ ideale. Allora si ha che:
  \begin{enumerate}
  \item $S^{-1}A \rar S^{-1}B$ è intera
  \item $\sfrac{A}{I^c} \rar \sfrac{B}{I}$ è intera
  \end{enumerate}
\item ({\bf Relazioni con estensioni di campi}) Supponiamo $A$ dominio e
  $K = \Frac(A)$ suo campo delle frazioni e consideriamo
  $A \subseteq K \subseteq L$ dove $\sfrac{L}{K}$ è algebrica. Definiamo
  $B = \intclos{A}{L}$.
  \begin{enumerate}
  \item Sia $x \in L$ e $\mu_x$ suo polinomio minimo su $K$. Se
    $\mu_x \in A[t]$ allora $x$ è intero su $A$.
  \item Se $A$ è normale vale anche il viceversa, ovvero si ha $x$ è
    intero su $A$ $\sse$ $\mu_x \in A[t]$.
  \end{enumerate}
\item ({\bf UFD $\implies$ normale}) Se $A$ è un UFD allora è normale.
\item ({\bf estensioni intere di campi}) Sia $A \subseteq B$
  un'estensione intera di domini. Allora $A$ è un campo $\sse$ $B$ è un
  campo. Ne segue che:
  \begin{enumerate}
  \item Sia $\kq \subseteq B$ un ideale. Allora $\kq$ è massimale $\sse$
    $\kq^c$ è massimale
  \item Se prendo $\kp$ primo di $A$ e $\kq_1, \kq_2 \in \Spec B$ tali
    che $\kq_1^c = \kq_2^c = \kq$ e $\kq_1 \subseteq \kq_2$ allora vale
    che $\kq_1 = \kq_2$.
  \end{enumerate}
\item ({\bf Lying Over}) Se $A \subseteq B$ è intera, allora $\forall
  \kp \in \Spec A \quad \exists \kq \in \Spec B$ tale che $\kq^c = \kp$,
  ovvero $f^*: \Spec B \rar \Spec A$ è surgettiva.
\item ({\bf Going Up}) Se $A \subseteq B$ è intera, allora ha la
  proprietà del going up, ovvero se
  $\forall \kp_1 \subseteq \kp_2 \subseteq A$ primi e
  $\forall \kq_1 \subseteq B$ primo tale che $\kq_1^c = \kp_1$ allora
  $\exists \kq_2 \supseteq \kq_1$ primo tale che $\kq_2^c = \kp_2$
\item ({\bf Going Down}) $A \subseteq B$ estensione intera di domini,
  con $A$ normale allora vale la proprietà del going down, ovvero
  $\forall \kp_1 \subseteq \kp_2 \subseteq A$ primi e
  $\forall \kq_2 \subseteq B$ primo tale che $\kq_2 \cap A = \kp_2$ si
  ha che $\exists \kq_1 \subseteq \kq_2$ primo tale che
  $\kq_1 \cap A = \kp_1$. \newline
  Da notare che serve sia la condizione di dominio che la normalità di
  $A$ per far funzionare tutto ciò.
\item Se $A$ è un dominio normale, $K = \Frac A$ ed $\sfrac{L}{K}$ è
  un'estensione algebrica di campi, $B = \intclos{A}{L}$,
  $I \subseteq A$ ideale, ed $x \in L$ allora $x$ è intero su $I$ $\sse$
  $\mu_{\sfrac{L}{K}, x} \in \sqrt{I}[t]$
\item ({\bf Simil-Galois}) Sia $A$ normale e $K = \Frac A \subseteq L$
  con $\sfrac{L}{K}$ estensione di Galois finita e $B = \intclos{A}{L}$
  e sia $G = \Gal \sfrac{L}{K}$. Allora si ha che:
  \begin{enumerate}
  \item $g(B) \subseteq B \quad \forall g \in G$
  \item Fissato un primo $\kp \in \Spec A$ si ha
    $\cF_\kp = \{ \kq \in \Spec B \pipe \kq \cap A = \kp \}$
    Allora $G$ agisce transitivamente su $\cF_\kp$
  \end{enumerate}
\item ({\bf Chiusura integrale di $A[x]$}) Sia $A \subseteq B$ e sia
  $C = \intclos{A}{B}$. Allora la chiusura integrale di $A[x]$ in $B[x]$
  è $C[x]$.
\item ({\bf Interezza e Nötherianità}) $A \subseteq B$ con $A$
  Nötheriano e $B$ finito su $A$. Allora $B$ è Nötheriano come
  anello. \newline
  Attenzione che $A \subseteq B$ con $A$ Nötheriano ed estensione intera
  NON implica che $B$ sia Nötheriano.
\item Sia $A$ dominio, $K = \Frac A$, $\sfrac{L}{K}$ un'estensione
  finita di campi e sia $B = \intclos{A}{L}$. Allora si ha:
  \begin{enumerate}
  \item È FALSO che se $A$ è Nötheriano allora $B$ lo sia.
  \item Se $A$ è normale ed $\sfrac{L}{K}$ è un'estensione separabile
    allora si ha che se $A$ è Nötheriano allora $B$ diventa un
    $A$-modulo finito e quindi è Nötheriano.
  \end{enumerate}
\item $A$ noetheriano e dominio, con $K = \Frac A \subseteq L$ campi e
  vorremmo poter dire qualcosa anche se $A$ non è normale. Ci sono due
  casi significativi nei quali si ha che in queste ipotesi
  $\intclos{A}{K}$ (la normalizzazione di $A$) è Nötheriana:
  \begin{enumerate}
  \item $\Dim A = 1$
  \item $A$ è una $K$-algebra finitamente generata
  \end{enumerate}
\item ({\bf Normalizzazione di Nöther}) $A$ $K$-algebra f.g. allora
  $\exists x_1, \ldots, x_n \in A$ algebricamente indipendenti tali che
  $A$ è finita (come modulo) su $K[x_1, \ldots, x_n]$
\item ({\bf Lemmi generici e fatti vari}) Le seguenti cose valgono:
  \begin{enumerate}
  \item $E$ campo e sia $A$ una $E$-algebra finitamente generata. Allora
    $\forall I \subseteq A$ si ha che
    $$\sqrt{I} = \bigcap_{\km \text{ massimali } \quad \km \supseteq I} \km$$
  \item $A$ e $B$ due $K$-algebre finitamente generate con $K$ campo
    algebricamente chiuso ed $A$ dominio. Se $f^*: \Max B \rar \Max A$ è
    suriettiva allora $f$ è iniettiva.
  \item Per un modulo sono proprietà locali (e massimali) essere
    piatto, essere normale, essere nullo.
  \item Per un modulo NON sono proprietà locali essere Noetheriano,
    essere dominio.
  \item Per una sequenza di moduli essere esatta in un punto è una
    proprietà locale (e massimale)
  \end{enumerate}
\end{itemize}

\section*{Teoria della Dimensione e Grado di Trascendenza}
\begin{itemize}
\item ({\bf Dimensione in estensioni intere}) Se $A \subseteq B$ è
  intera allora $\Dim A = \Dim B$ (è compreso il caso in cui entrambe le
  dimensione siano infinite)
\item ({\bf Dimensione delle $K$-algebre f.g.}) Sia $K$ un campo e
  $f \in K[x_1, \ldots, x_n]$ con $f \neq 0$. Allora si ha
  $\Dim K[x_1, \ldots, x_n]_f = n$
\item ({\bf Relazione con il grado di trascendenza}) Sia $A$ $K$-algebra
  f.g. e $A$ dominio. Allora $\Dim A = \trdeg_K A$
\item ({\bf Cardinalità di una base di trascendenza}) $A$ dominio e
  $K$-algebra f.g. Allora tutte le basi di trascendenza hanno la stessa
  cardinalità. \newline
  Inoltre, se $A$ è un dominio, $E$ una $K$-algebra e sia $L = \Frac A$
  \begin{enumerate}
  \item $x_i \in A$ è una base di trascendenza di $A$ $\sse$ lo è di $L$
  \item Se $y_1, \ldots, y_n$ è base di trascendenza di $L$ su $K$
    allora $\exists b \in A$ tali che $by_1, \ldots, by_n$ è una base di
    trascendenza di $A$.
  \end{enumerate}
\item ({\bf Particolarità delle $K$-algebre}) Per le $K$-algebre
  f.g. valgono le seguenti cose:
  \begin{enumerate}
  \item Sono anelli catenari.
  \item Se $A$ è dominio, preso un primo $\kp$ di altezza uno si ha
    $\Dim \sfrac{A}{\kp} = \Dim A - 1$
  \item $\kp \in \Spec A \implies $ $\Ht \kp, \Coht p < + \infty$
  \item Se $A$ è dominio si ha $\forall \kp \in \Spec A$ vale che
    $\Dim A = \Ht \kp + \Coht \kp$
  \end{enumerate}
\item ({\bf Artinianità e Nötherianità}) $A$ artiniano se e solo se $A$
  Nötheriano e di dimensione zero.
\item ({\bf Richiami di Decomposizione Primaria}) Sia $I \subseteq A$ un
  ideale. Una decomposizione primaria di $I$ è una scrittura
  $I = \cap_i Q_i$ dove i $Q_i$ sono un numero finito di ideali
  primari. Se $A$ è Nötheriano valgono:
  \begin{enumerate}
  \item Primi associati ad $I$: $\Ass I = \{ \kp \in \Spec A \pipe \exists x
    \in A \tc \kp = (I : x) \}$
  \item Gli zero divisori di $A$ sono l'unione dei primi associati a
    zero: $$\cD(A) = \cup_{\kp \in \Ass 0} \kp$$
  \item $\Ass_{S^{-1}A} S^{-1}I = \Ass_{A} I \cap \Spec S^{-1}A$
  \item $\Ass I$ è finito
  \item Se $\kp$ è minimale sopra $I$, allora $\kp$ è associato ad $I$.
  \end{enumerate}

\item ({\bf Esercizi e Lemmi vari}) Valgono le seguenti cose:
  \begin{enumerate}
  \item $\prod_{n = 0}^\infty \bbZ$ NON è uno $\bbZ$-modulo libero
  \item $\prod_{n = 0}^\infty \rar^f \bbZ$ tale che $f(e_i) = 0 \quad
    \forall i$ allora deve essere che $f = 0$
  \item $X = \Spec A$ e
    $X_f := \{ \kp \subseteq A \pipe f \notin \kp \} = X \setminus V(f)$ è
    un aperto di $X$. Questi sono un sistema fondamentale di aperti di
    $X$ e si ha $X_f \cong \Spec A_f$
  \item $K((t))$ NON è algebrico su $K(t)$
  \item $X = \Spec A$ è compatto, qualunque sia $A$.
  \item $A \subseteq B$ intera $\implies f^*$ chiusa
  \item $f^*$ chiusa $\implies$ vale il Going Up.
  \end{enumerate}
\end{itemize}

\section*{Dimensione e Anelli Graduati}
\begin{itemize}
\item ({\bf Serie di Jordan-Hölder}) $A$ anello c.u., $M$ un
  $A$-modulo. Una serie di Jordan-Hölder per $M$ è una successione
  crescente di sottomoduli
  $0 = M_1 \subsetneq \ldots \subsetneq M_n = M$ tali che
  $\sfrac{M_i}{M_{i-1}}$ è un $A$-modulo semplice (ovvero è diverso dal
  modulo nullo e non ha sottomoduli propri)
\item ({\bf Lunghezza di un Modulo}) Se $M$ ha una serie di JH, diciamo
  che la lunghezza di $M$ è finita e definiamo
  $\cl(M) = \min \{ n \pipe \exists \text{ serie di JH con } n+1 \text{
    termini } \}$
\item ({\bf Lunghezza delle serie di JH}) Tutte le serie di JH di uno
  stesso modulo hanno la stessa lunghezza ed inoltre i fattori
  $\sfrac{M_i}{M_{i-1}}$ sono uguali per ogni serie, a meno di
  permutazioni.
\item ({\bf Comportamento per sequenze esatte}) Sia data una sequenza
  esatta corta di $A$-moduli $0 \rar X \rar Y \rar Z \rar 0$. Allora
  vale che:
  \begin{enumerate}
  \item $\cl(X), \cl(Z) < +\infty \sse \cl(Y) < +\infty$
  \item $\cl(Y) = \cl(X) + \cl(Z)$
  \end{enumerate}
  Inoltre si ha che per un generico modulo $M$ vale $\cl(M) < \infty
  \sse M$ è artiniano e Nötheriano.
\item ({\bf Anelli graduati noetheriani}) $A$ anello graduato, allora
  $A$ è noetheriano se e solo se $A_0$ è noetheriano ed $A$ è f.g. come
  $A_0$-algebra.
\item ({\bf Funzione e Serie di Hilbert}) Preso $A$ graduato, $A_0$
  artiniato, $A$ noetheriano, $M$ graduato e f.g. (ovvero $M$
  noetheriano) definiamo:
  \begin{enumerate}
  \item $n \mapsto \cl_{A_0}(M_n)$ funzione di Hilbert
  \item $P_M(t) := \sum_n t^n \cl_{A_0}(M_n)$ serie di Hilbert
  \end{enumerate}
  Inoltre in queste ipotesi se si ha una sequenza esatta corta con
  morfismi graduati di grado zero:
  $0 \rar X \rar^\phi Y \rar^\psi Z \rar 0$ allora sono definiti i
  polinomi di hilbert e vale che $P_Y = P_X + P_Z$
\item ({\bf Teorema di Hilbert}) $A$ graduato, $M$ graduato, $A_0$
  artiniano e $A$ noetheriano, $M$ f.g. e si chiamino $a_1, \ldots, a_k$
  i generatori omogenei di $A$ come $A_0$-algebra e $d_i := \Deg
  a_i$. Allora $\exists f \in \bbZ[t, t^{-1}]$ polinomio di Laurent tale
  che
  $$P_M(t) = \frac{f(t)}{\prod_{i=1}^k (1 - t^{d_i})}$$
\item ({\bf funzione di Hilbert e grado}) Se $A$ è generato in grado $1$
  allora la funzione di Hilbert (nelle ipotesi precedenti) per $n$
  grandi coincide con i valori assundi da un polinomio di grado $d(M) -
  1$ (dove $d(M)$ è l'ordine di polo in $1$ della funzione razionale
  $P_M$)
\item ({\bf Analogo di Nakyama}) $A$ graduato, $M$ graduato f.g. e
  supponiamo che $A_+ M = M$ allora $M = 0$

\end{itemize}

\end{document}

